\chapter{Analisi}

\section{Procedura}
%L'attività di analisi iniziale, ovvero senza l'ausilio di librerie specifiche per l'invio di determinati pacchetti, è stata effettuata installando un server Apache sia su Windows 11 che su Kali.
%Si è quindi proceduto ad inviare richieste ai server tramite browser web e all'analisi dei pacchetti ricevuti in risposta con l'utilizzo di Wireshark.
%Successivamente sono state analizzate anche le risposte a determinati pacchetti inviati utilizzando la libreria Scapy.
%Di seguito sono riportate le differenze più importanti rilevate durante l'intera attività di analisi.
L'analisi dei pacchetti per l'individuazione delle differenze è stata effettuata sulle risposte ricevute da server Apache installati su Windows 11 e su Kali.
L'attività si analisi è stata svolta in due differenti modalità:
\begin{enumerate}
	\item Nella prima sono state analizzate le risposte ricevute in seguito a richieste GET HTTP inviate tramite browser web; la cattura e l'osservazione di queste è stata effettuata tramite Wireshark.
	\item Nella seconda si è proceduto con l'ispezione delle risposte a specifici pacchetti inviati utilizzando la funzione sr1 della libreria Python Scapy, che permette la cattura delle risposte ai pacchetti inviati. Le risposte sono state esaminate con la funzione show.
\end{enumerate}
Di seguito vengono mostrate le principali differenze notate tra i due sistemi operativi durante l'intera attività di analisi.

\section{Livello 3: protocolli IP e ICMP}
È stata effettuata in primis l'analisi del TTL (\textit{Time To Live}), essendo un valore estremamente semplice da analizzare.
Questo campo contiene un numero intero positivo che viene decrementato ad ogni \textit{hop}, ovvero ad ogni router che il pacchetto incontra nel percorso verso l'host ricevente, e viene eliminato dalla rete quando questo raggiunge lo 0.
Il protocollo, però, non impone nessun valore di partenza, lasciando libertà di scelta al sistema operativo del mittente; l'unico limite è rappresentato dagli 8 bit riservati a quel campo (ovvero ad un valore massimo di 255).
L'analisi ha portato ad evidenziare la seguente differenza riguardo al TTL:
\begin{table}[H]
	\centering
	\begin{tabular}{| l | c |}
		\hline
		\rowcolor{blue!10} Windows 11 & 128
		\\
		\hline
		\rowcolor{red!10} Kali & 64
		\\
		\hline
		
	\end{tabular}
	\caption{TTL iniziale}
	\label{tab:TTL}
\end{table}

Questa differenza è molto importante in quanto è presente in ogni pacchetto inviato in rete e, se non vengono apportate modifiche, non varia tra un pacchetto e l'altro. Solitamente i valori iniziali sono potenze di due (64 o 128); l'eventuale utilizzo di valori parecchio inusuali denota un comportamento molto riconoscibile.
\\
\\
Inoltre, è stata effettuata l'analisi del protocollo ICMP, e in particolare è stata notata una differenza nel campo \textit{code} in caso di risposta a determinati pacchetti inviati utilizzando la libreria di Python scapy.

\begin{lstlisting}[language=Python, caption={Comando Python per l'invio del pacchetto}]
	from scapy.all import *
	
	pkt = sr1(IP(dst='192.168.63.1')/ICMP(code=9))
	pkt.show()
	
\end{lstlisting}
Questo comando invia un pacchetto ICMP con il campo \textit{code} impostato ad un numero casuale, in questo caso 14: in questo specifico contesto, Windows 11 risponde inviando un pacchetto in cui \textit{code}=0, mentre Kali copia il valore che ha ricevuto nella richiesta.

\begin{table}[h]
	\centering
	\begin{tabular}{ | l | c |}
		\hline
		\rowcolor{blue!10} Windows 11 & 0
		\\
		\hline
		\rowcolor{red!10} Kali & Stesso valore inviato nella richiesta
		\\
		\hline

	\end{tabular}
	\caption{Campo code quando nella richiesta è diverso da 0}
	\label{tab:code}
\end{table}

La differenza è molto peculiare perchè riconoscibile utilizzando solamente il protocollo ICMP (lo stesso del comando \textit{ping}) e non influenza il normale utilizzo di questo protocollo dal momento in cui la differenza si presenta solo inviando determinati pacchetti.
\section{Livello 4: analisi protocollo TCP}
Il livello 4 dello stack è formato da due protocolli: User Datagram Protocol (UDP) e Transmission Control Protocol (TCP). Entrambi forniscono informazioni utili per il fingerprinting, ma il TCP possiede un numero maggiore di campi e di opzioni utilizzabili, pertanto è stata data priorità all'analisi di quest'ultimo.

Analizzando l'handshake TCP, si può notare una discrepanza nell'utilizzo del Window Scale; quest'opzione consente di aumentare la Window Size oltre il valore che si otterrebbe settando tutti i bit di quel campo a 1.
Ciò è dovuto al fatto che il valore dell'opzione rappresenta il numero di shift verso sinistra dei bit del Window Size, che corrispondono a raddoppi del valore contenuto. 
L'utilizzo di questa opzione viene concordato nei segmenti SYN e SYN+ACK dell'handshake e il suo valore non viene più modificato per il resto della connessione. Confrontando i valori ottenuti da Windows 11 e Kali, si ottiene il seguente risultato:
\\
\begin{table}[htb]
	\centering
	\begin{tabular}{| l | c |}
		\hline
		\rowcolor{blue!10} Windows 11 & 8
		\\
		\hline
		\rowcolor{red!10} Kali & 7
		\\
		\hline
		
	\end{tabular}
	\caption{Window Scaling Factor}
	\label{tab:Window Scale}
\end{table}

Si tratta di una differenza importante, la quale però per essere rilevata necessita del livello 4 dello stack; inoltre, si può osservare solo se si utilizza il protocollo TCP, essendo UDP privo di handshake.
\\
\\
Proseguendo l'analisi si possono notare ulteriori differenze riguardanti il flag sulla notifica esplicita di congestione (ECN). Si tratta di un flag che consente di comunicare a livello end to end una congestione, evitando però la perdita dei pacchetti. Il mittente, infatti, se riceve un pacchetto contenente quest'informazione diminuisce i pacchetti inviati in quella comunicazione, seguendo specifici algoritmi (ad esempio, Reno). 
Se si invia un pacchetto simulando l'inizio di un handshake (SYN) con i flag sulla congestione attivi e si osserva la risposta (SYN+ACK), si possono notare risultati differenti tra i pacchetti ricevuti da Windows 11 e Kali:
\\
\begin{lstlisting}[language=Python, caption={Comando Python per l'invio del pacchetto}]
	from scapy.all import *
	
	pkt=sr1(IP(dst='192.168.63.1')/TCP(dport=80, flags='SCE'))
	pkt.show()
	
\end{lstlisting}

\begin{table}[h]
	\centering
	\begin{tabular}{| l | c |}
		\hline
		\rowcolor{blue!10} Windows 11 & 0
		\\
		\hline
		\rowcolor{red!10} Kali & 1
		\\
		\hline
		
	\end{tabular}
	\caption{ECN in risposta a specifico pacchetto}
	\label{tab:ECN}
\end{table}

Questa diversità è molto peculiare per il fingerprinting di Nmap che, come verrà meglio specificato nella sezione \ref{algoritmi}, assegna una punteggio elevato al risultato di questo test. 

\section{Livello 7: analisi protocollo HTTP}
Sebbene HTTP sia un protocollo a livello applicativo, esso consente ugualmente di ricavare alcune informazioni utili per l'OS fingerprinting: il campo \textit{User-Agent}, infatti, contiene informazioni esplicite riguardanti il browser che si sta utilizzando e il sistema operativo in utilizzo. Questo tipo di situazione prende il nome di \textit{banner grabbing}.

Inoltre, a questo livello dello stack si può tentare un fingerprinting che vada oltre l'individuazione del sistema operativo, ponendo come obiettivo quello di indovinare il tipo di applicativo in uso.
Questo è reso possibile dal fatto che HTTP è un protocollo di tipo testuale, pertanto non vi sono campi prefissati in determinati bit per ogni funzione.
L'header, infatti, contiene varie coppie secondo lo schema:\\
\textbf{chiave:valore} 


Questo, a differenza dei protocolli analizzati precedentemente, permette un diverso ordine con le quali le coppie vengono elencate, essendo questo ininfluente per una corretta comunicazione; ne consegue che analizzare l'ordinamento è molto importante se si vuole tentare di individuare, ad esempio, il tipo di client in utilizzo.\\

Osservando i pacchetti HTTP inviati dai principali browser web si possono notare delle differenze sull'ordinamento per quanto riguarda Firefox e altri browser basati su Chromium.
L'ordine dei campi nell'header è infatti il seguente:

\begin{table}[h]
	\centering
	\begin{tabular}{| c | c |}
		\hline
		\textbf{Firefox} & \textbf{Chromium-based}
		\\
		\hline
		Host & Host
		\\
		\hline
		User-Agent & Connection
		\\
		\hline
		Accept & Upgrade-Insecure-Requests
		\\
		\hline
		Accept-language & User-Agent
		\\
		\hline
		Accept-Encoding & Accept
		\\
		\hline
		Connection & Accept-Encoding
		\\
		\hline
		Upgrade-Insecure-Requests & Accept-Language
		\\
		\hline
	\end{tabular}
	\caption{Ordine campi header HTTP}
	\label{tab:ordineHTTP}
\end{table}

Inoltre, analizzando i linguaggi accettati si possono osservare ulteriori disuguaglianze riguardanti il campo \textit{Quality Value}.
Si tratta di un valore indicato tramite la lettera q (case insensitive) che esprime la preferenza a determinati linguaggi tramite un numero compreso tra 0 e 1, avente massimo tre cifre decimali; il valore di default è 1 ovvero la massima preferenza. \cite{qvalue} 

Segue la stringa di quello specifico campo riguardante Chrome e Firefox:

\begin{lstlisting} [caption={Campo Accept-Language di Chrome}]
	Acceept-Language: it-IT,it;q=0.9,en-US;q=0.8,en;q=0.7
\end{lstlisting}

\begin{lstlisting}[caption={Campo Accept-Language di Firefox}]
	Acceept-Language: it-IT,it;q=0.8,en-US;q=0.5,en;q=0.3
\end{lstlisting}

Si possono dunque distinguere delle differenze tra i due browser, che ovviamente possono rivestire un ruolo importante in fase di fingerprinting dello User-Agent utilizzato.
\\

È possibile, inoltre, osservare ulteriori differenze che vi sono tra i due browser analizzando il campo \textit{Accept} dell'Header HTTP. Si osservino infatti i due campi nei due browser web menzionati precedentemente:
\\


\begin{lstlisting}[caption={Campo \textit{Accept} di richiesta GET di Chrome}]
	text/html,application/xhtml+xml,application/xml;q=0.9,image/avif,
	image/webp,image/apng,*/*;q=0.8,application/signed-exchange;v=b3;q=0.9
\end{lstlisting}

\begin{lstlisting}[caption={Campo \textit{Accept} di richiesta GET di Firefox}]
	text/html,application/xhtml+xml,application/xml;q=0.9,image/avif,
	image/webp,*/*;q=0.8
\end{lstlisting}
















