\chapter{Teoria}

\section{Stack TCP/IP}
Per effettuare comunicazioni tramite internet, vi è il bisogno che tutti i dispositivi connessi rispettino determinati meccanismi; questo si rende necessario a causa dell'elevata eterogeneità derivata da hardware e software differenti.
Questi meccanismi, che prendono il nome di \textit{protocolli}, sono strutturati secondo diversi layer (livelli) formando lo stack TCP/IP.  \\
Sebbene l'idea originale prevedesse un modello composto da sette livelli, de facto lo schema attualmente in uso ne prevede solamente quattro. Nonostante ciò, nella terminologia informatica la numerazione dei livelli è rimasta quella precedente.
\\
\begin{table}[htb]
	\centering
	\begin{tabular}{| l | c |}
		\hline
		Livello 7 & Applicativo
		\\
		\hline
		Livello 4 & Trasporto
		\\
		\hline
		Livello 3 & Rete
		\\
		\hline
		Livello 2 & Fisico
		\\
		\hline
		
	\end{tabular}
	\caption{Livelli dello stack TCP/IP}
	\label{tab:stack}
\end{table}

\subsection{Funzionamento dello stack TCP/IP}
Il meccanismo dello stack prevede che ad ogni livello vengano aggiunti al messaggio delle intestazioni (header), che verranno valutate dal rispettivo livello del ricevente.
Si precisa che l'ordine in cui si valutano gli header dei livelli è inverso rispetto a quello del mittente; chi invia partirà dal livello più alto, mentre chi riceve dal più basso.
Questa procedura prende il nome di \textit{incapsulamento} ed è riassunta nella seguente figura:


\begin{figure}[h]
	\includegraphics[width=\textwidth]{figures/incapsulamento.png}
	\caption{METTERE RIFERIMENTO IMMAGINE: http://infodoc.altervista.org/sistemi-e-reti/incapsulamento/}
	\label{incapsulamento}
\end{figure}

\subsection{Header per il fingerprinting}
Gli header aggiunti ad ogni livello sono formati da vari campi contententi informazioni utili per la comunicazione, e il valore che questi assumono in determinate situazioni è dipendente dal sistema operativo che si sta utilizzando.

Si prenda ad esempio l'header del protocollo TCP:\\

\begin{figure}[H]
	\includegraphics[width=\textwidth]{figures/headerTCP.JPG}
	\caption{METTERE RIFERIMENTO IMMAGINE: https://www.ionos.it/digitalguide/server/know-how/presentazione-tcp/}
	\label{headerTCP}
\end{figure}

Il campo \textit{option} permette di segnalare al ricevente l'uso di alcune opzioni di comunicazione, come il window scaling; il loro supporto e l'effettivo utilizzo, essendo queste facoltative e quindi peculiari di specifici sistemi operativi, rivestono quindi particolare importanza ai fini del fingerprinting.
Esempi analoghi si possono trovare nei protocolli di tutti i livelli dello stack, e l'unione delle informazioni acquisite consente di poter individuare con una discreta precisione il sistema operativo del dispositivo target.