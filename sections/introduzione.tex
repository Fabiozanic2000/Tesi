

\chapter{Introduzione}
\label{introduzione}

\section{OS Fingerprinting}
\label{citazioni}

L'OS fingerprinting consiste nel rilevare da remoto il sistema operativo di un dispositivo analizzandone i pacchetti inviati. Le differenze di implementazione dello stack TCP/IP, infatti, determinano comportamenti diversi che, analizzati, consentono di ottenere informazioni importanti. \\
Il fingerprinting può essere effettuato in due modalità: attiva e passiva. \\ 
Nella prima si analizzano le risposte ricevute in seguito ad alcuni pacchetti inviati; questi ultimi sono appositamente costruiti in modo da massimizzare le informazioni che si possono ottenere dalla risposta.
Nella seconda, invece, viene ispezionato il normale traffico del dispositivo target; si tratta quindi di una tecnica meno invasiva e che si espone meno al rischio di essere scoperti.

\section{Stack TCP/IP}
Per effettuare comunicazioni tramite internet, vi è il bisogno che tutti i dispositivi connessi rispettino determinati meccanismi; questo si rende necessario a causa dell'elevata eterogeneità derivata da hardware e software differenti.
Questi meccanismi, che prendono il nome di \textit{protocolli}, sono strutturati secondo diversi layer (livelli) formando lo stack TCP/IP.  \\
Sebbene l'idea originale prevedesse un modello composto da sette livelli, de facto lo schema attualmente in uso ne prevede solamente quattro. Nonostante ciò, nella terminologia informatica la numerazione è rimasta quella precedente

\begin{table}[htb]
	\centering
	\begin{tabular}{| l | c |}
		\hline
		Livello 7 & Applicativo
		\\
		\hline
		Livello 4 & Trasporto
		\\
		\hline
		Livello 3 & Rete
		\\
		\hline
		Livello 2 & Fisico
		\\
		\hline

	\end{tabular}
	\caption{Livelli dello stack TCP/IP}
	\label{tab:stack}
\end{table}

\subsection{Funzionamento dello stack TCP/IP}
Si consideri l'esempio dell'invio di una lettera tramite il servizio di poste. La procedura da seguire è la seguente:

\begin{enumerate}
	\item Il mittente scrive il contenuto del messaggio su un foglio e successivamente lo inserisce nella busta.
	\item Il mittente scrive, nella busta, il nominativo del destinatario.
	\item Il mittente scrive il CAP e l'indirizzo del ricevente.
	\item Il mittente, dopo aver incollato il francobollo, consegna la busta al servizio di poste.
\end{enumerate}

Si noti che la sequenza di eventi che si verifica per la ricezione delle lettere è nell'ordine opposto a quello precedente:
\begin{enumerate}
	\item Viene controllata la presenza del francobollo.
	\item Il postino legge l'indirizzo del destinatario e consegna la lettera.
	\item Verrà letto il nominativo per individuare a quale inquilino è diretta.
	\item Il destinatario apre la busta e legge il contenuto del messaggio.
\end{enumerate}

La sequenza di eventi descritta rappresenta ciò che avviene anche quando si comunica tramite internet. Ad ogni livello dello stack, infatti, vi è un protocollo che aggiunge una sua intestazione (\textit{header}) al pacchetto del mittente, ad iniziare dal livello più alto disponibile per quel dispositivo e il ricevente  valuterà queste in ordine inverso. Ad ogni layer il protocollo utilizzato può essere differente; fondamentale, ovviamente, che questo sia supportato da entrambi gli attori della conversazione.\\
Questa metodologia prende il nome di \textit{incapsulamento}.
\\

\begin{figure}
	\includegraphics[width=\textwidth]{figures/incapsulamento.png}
	\caption{METTERE RIFERIMENTO IMMAGINE: http://infodoc.altervista.org/sistemi-e-reti/incapsulamento/}
	\label{incapsulamento}
\end{figure}



Per valutare il sistema operativo di un determinato dispositivo, si fa affidamento agli header di ogni livello.






\section{Strumenti per il fingerprinting}
Esistono numerosi tool per effettuare il fingerprinting, che eseguono anche attività correlate e fondamentali per quest'ultimo come ad esempio il port scanning.
Nel corso di questo documento si farà principalmente riferimento a due strumenti:
\begin{itemize}
	\item Nmap, per il fingerprinting attivo
	\item p0f, per il fingerprinting passivo
\end{itemize}


	






