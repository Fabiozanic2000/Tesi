

\chapter{Introduzione}
\label{introduzione}

L'OS fingerprinting consiste nel rilevare da remoto il sistema operativo di un dispositivo analizzandone i pacchetti inviati. Le differenze di implementazione dello stack TCP/IP, infatti, determinano comportamenti diversi che analizzati consentono di ottenere informazioni utili a questo scopo. \\
Il fingerprinting può essere effettuato in due modalità: attiva e passiva. \\ 
Nella prima si analizzano le risposte ricevute in seguito ad alcuni pacchetti inviati appositamente costruiti allo scopo di massimizzare le informazioni che si possono ottenere dalla risposta.
Nella seconda, invece, viene ispezionato il normale traffico del dispositivo target; si tratta quindi di una tecnica meno invasiva e che si espone meno al rischio di essere scoperti grazie all'assenza di pacchetti inviati.
\\
\\
L'obiettivo di questa tesi consiste nell'analizzare le differenze che portano all'individuazione del sistema operativo, e successivamente modificare determinati parametri di quest'ultimo in modo da riuscire a ingannare i principali strumenti per il fingerprinting.
I risultati ottenuti, quindi, dovranno essere errati e portare all'individuazione di un sistema operativo differente rispetto a quello realmente in uso.\\
Una delle ragioni principali che porta all'individuazione di un sistema operativo da remoto è che quando viene effettuato un attacco su quel sistema (da un hacker oppure dall'amministratore di sistema per una verifica della sicurezza), l'attaccante potrebbe avere un solo tentativo a disposizione \footnote{Riaan Stopforth,\textit{ Techniques and Countermeasures ofTCP/IP OS Fingerprinting on Linux Systems}.}. La sua individuazione può quindi risultare cruciale ai fini della sicurezza.
Per l'ottenimento dell'offuscamento, si è proceduto in primis con l'identificazione delle differenze tra i sistemi analizzando i pacchetti che venivano inviati in risposta a richieste GET HTTP; a tale scopo sono stati installati server Apache su entrambi i sistemi operativi osservati. I tentativi di offuscamento sono stati effettuati tramite modifiche ad alcuni file di configurazione e la manipolazione di pacchetti in uscita.
\\
Infine, sono state valutate le differenze tra i principali browser web rispetto all'handshake TLS; l'analisi di quest'ultimo può infatti portare all'individuazione del browser in utilizzo.










